\documentclass[12pt,letter]{article}
\usepackage{geometry}\geometry{top=0.75in}
\usepackage{amsmath}
\usepackage{amssymb}
\usepackage{mathtools}
\usepackage{xcolor} % Color words
\usepackage{cancel} % Crossing parts of equations out
\usepackage{tikz}       % Drawing 
\usepackage{pgfplots}   % Other plotting
\usepgfplotslibrary{colormaps,fillbetween}
\usepackage{placeins}   % Float barrier
\usepackage{hyperref}   % Links
\usepackage{tikz-qtree} % Trees
\usepackage{graphicx}
\usepackage{subcaption}
\usepackage{multicol}
\usepackage{graphicx}   % For graphics
\usepackage{parcolumns}
\usepackage{listings}   % lstlisting
\usepackage{pdfpages}

\begin{document}
\title{CIS 510: Final Project Proposal}
\author{Steven Walton}
\maketitle

In this final project I will use my research in Higher Order Lagrange Elements
and demonstrate their uses in VTK-m. A minimum viable product is a demonstration
of a mesh and field being read into VTK-m. I can also demonstrate linearized
versions of higher order hexahedrons.

If this does not work out my backup is to use my DRP machine learning research.
The minimum viable product here is demonstrating my current progress and
discussing the challenges and different models that I have used to mixed
success.

\end{document}
