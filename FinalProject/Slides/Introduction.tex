\begin{frame}
\frametitle{Overview}
    \begin{itemize}
        \item \emph{\color{UOYellow}Background}
        \item Methodology 
        \item Results
        \item Future Work
    \end{itemize}
\end{frame}

\begin{frame}
\frametitle{Background}
    \begin{itemize}
        \item Growth in compute speed in supercomputers is out pacing growth in
            IO speed and storage by 10 to 1.
        \item Supercomputers are generating so much data so fast that we cannot
            save it fast enough and do not have enough storage to save high
            resolution time series data.
        \item Simulation science is exploratory by nature. We don't always know
            what we want a priori (beforehand)
        \item Interesting events can happen between save states.
    \end{itemize}
\end{frame}

\begin{frame}
\frametitle{What We Need}
    \begin{itemize}
        \item Need to be able to explore the simulation in situ.
        \item Need to be able to extract information across large time steps
            while saving data infrequently. 
    \end{itemize}
\end{frame}

\begin{frame}
    \frametitle{What Has Been Done}
    \begin{itemize}
        \item Conventionally scientists LERP between timesteps.
        \item Current research is into different machine learning models to
            infer data between timesteps. 
    \end{itemize}
\end{frame}
